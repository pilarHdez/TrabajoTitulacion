\documentclass[../reportesINE.tex]{subfiles}

\begin{document}

\begin{center}
\textit{“Existen dos formas de desarrollar un diseño de software: Una es hacerla tan simple que obviamente no hay deficiencias, y la otra es que sea tan complicada que no existan deficiencias obvias. El primer método es mucho más difícil.”} \\
- C.A.R. Hoare
\end{center}

En enero del 2016, se solicitó la construcción e implementación del \textit{Sistema Integral de consulta} como apoyo a la \textit{Dirección Jurídica} en la gestión y control integral de los servicios que presta a los órganos del \textit{INE}. El proyecto fue desarrollado de acuerdo a metodologías o modelos que actualmente existen en el \textit{INE}, asegurando la inclusión de todos y cada uno de los requerimientos funcionales y no funcionales. 
\\ \\
\textit{La Dirección Jurídica} presenta una necesidad imperiosa por establecer registro, control, mecanismos eficientes y efectivos para obtener información, de los procesos que tiene a su cargo. Además en la \textit{Dirección Jurídica} operan cinco aplicaciones, algunas de estas proveen información a otros órganos del Instituto o instituciones externas. La explotación de esta información es de vital importancia para la toma de decisiones. 
\\ \\
El sistema debe proporcionar información precisa, oportuna, completa y ordenada sobre los datos que se generen de las actividades que integran cada uno de los procesos de la \textit{Dirección Jurídica}. Por lo que la gobernabilidad de datos, es necesaria para garantizar la entrega de datos fiables y seguros.
\\ \\
Para cumplir con este requerimiento, el \textit{Sistema Integral de la Dirección Jurídica} cuenta la funcionalidad para obtener informes y reportes de indicadores, estadísticas que faciliten acciones proactivas y también integra consultas de información actualizada, tanto para su publicación como para la toma de decisiones y permite la exportación de información a formatos como Excel o PDF. 
\\ \\
Esta funcionalidad se encuentra disponible en un módulo de Reportes, al cual se tiene acceso con un rol de tipo consulta, se realizó en un proceso que comprede cuatro etapas: 

\begin{itemize}
 \item Análisis de los requerimientos: Se definió lo más claramente posible el problema a resolver. \\
 \item Diseño: Una vez que la información estuvo recolectada se desarrollaró un modelo o las especificaciones para el producto. \\
 \item Generación de código: Se realizó el desarrollo del módulo en el lenguaje de programación Java, con Spring y PrimeFaces. \\
 \item Pruebas y funcionamiento: Se verificó que el módulo desarrollado cumpliera con los requerimientos de la especificación creada durante la etapa de diseño.
\end{itemize}


El desarrollo de dicho módulo me fue encomendado, en las siguientes página expongo a detalle cada uno de los pasos que fueron necesarios para su implementación. 

\end{document}